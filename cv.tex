%-----------------------------------------------------------------------------------------------------------------------------------------------%
%	The MIT License (MIT)
%
%	Copyright (c) 2021 Jitin Nair
%
%	Permission is hereby granted, free of charge, to any person obtaining a copy
%	of this software and associated documentation files (the "Software"), to deal
%	in the Software without restriction, including without limitation the rights
%	to use, copy, modify, merge, publish, distribute, sublicense, and/or sell
%	copies of the Software, and to permit persons to whom the Software is
%	furnished to do so, subject to the following conditions:
%	
%	THE SOFTWARE IS PROVIDED "AS IS", WITHOUT WARRANTY OF ANY KIND, EXPRESS OR
%	IMPLIED, INCLUDING BUT NOT LIMITED TO THE WARRANTIES OF MERCHANTABILITY,
%	FITNESS FOR A PARTICULAR PURPOSE AND NONINFRINGEMENT. IN NO EVENT SHALL THE
%	AUTHORS OR COPYRIGHT HOLDERS BE LIABLE FOR ANY CLAIM, DAMAGES OR OTHER
%	LIABILITY, WHETHER IN AN ACTION OF CONTRACT, TORT OR OTHERWISE, ARISING FROM,
%	OUT OF OR IN CONNECTION WITH THE SOFTWARE OR THE USE OR OTHER DEALINGS IN
%	THE SOFTWARE.
%	
%
%-----------------------------------------------------------------------------------------------------------------------------------------------%

%----------------------------------------------------------------------------------------
%	DOCUMENT DEFINITION
%----------------------------------------------------------------------------------------

% article class because we want to fully customize the page and not use a cv template
\documentclass[a4paper,12pt]{article}

%----------------------------------------------------------------------------------------
%	FONT
%----------------------------------------------------------------------------------------

% % fontspec allows you to use TTF/OTF fonts directly
% \usepackage{fontspec}
% \defaultfontfeatures{Ligatures=TeX}

% % modified for ShareLaTeX use
% \setmainfont[
% SmallCapsFont = Fontin-SmallCaps.otf,
% BoldFont = Fontin-Bold.otf,
% ItalicFont = Fontin-Italic.otf
% ]
% {Fontin.otf} 

%----------------------------------------------------------------------------------------
%	PACKAGES
%----------------------------------------------------------------------------------------
\usepackage{url}
\usepackage{parskip} 	

%other packages for formatting
\RequirePackage{color}
\RequirePackage{graphicx}
\usepackage[usenames,dvipsnames]{xcolor}
\usepackage[scale=0.9]{geometry}

%tabularx environment
\usepackage{tabularx}

%for lists within experience section
\usepackage{enumitem}

% centered version of 'X' col. type
\newcolumntype{C}{>{\centering\arraybackslash}X} 

%to prevent spillover of tabular into next pages
\usepackage{supertabular}
\usepackage{tabularx}
\newlength{\fullcollw}
\setlength{\fullcollw}{0.47\textwidth}

%custom \section
\usepackage{titlesec}				
\usepackage{multicol}
\usepackage{multirow}

%CV Sections inspired by: 
%http://stefano.italians.nl/archives/26
\titleformat{\section}{\Large\scshape\raggedright}{}{0em}{}[\titlerule]
\titlespacing{\section}{0pt}{10pt}{10pt}

%for publications
\usepackage[style=authoryear,sorting=ynt, maxbibnames=2]{biblatex}

%Setup hyperref package, and colours for links
\usepackage[unicode, draft=false]{hyperref}
\definecolor{linkcolour}{rgb}{0,0.2,0.6}
\hypersetup{colorlinks,breaklinks,urlcolor=linkcolour,linkcolor=linkcolour}
\addbibresource{citations.bib}
\setlength\bibitemsep{1em}

%for social icons
\usepackage{fontawesome5}

%debug page outer frames
%\usepackage{showframe}

%----------------------------------------------------------------------------------------
%	BEGIN DOCUMENT
%----------------------------------------------------------------------------------------
\begin{document}

% non-numbered pages
\pagestyle{empty} 

%----------------------------------------------------------------------------------------
%	TITLE
%----------------------------------------------------------------------------------------

% \begin{tabularx}{\linewidth}{ @{}X X@{} }
% \huge{Your Name}\vspace{2pt} & \hfill \emoji{incoming-envelope} email@email.com \\
% \raisebox{-0.05\height}\faGithub\ username \ | \
% \raisebox{-0.00\height}\faLinkedin\ username \ | \ \raisebox{-0.05\height}\faGlobe \ mysite.com  & \hfill \emoji{calling} number
% \end{tabularx}

% \begin{tabularx}{\linewidth}{@{} C @{}}
% \Huge{Your Name} \\[7.5pt]
% \href{https://github.com/username}{\raisebox{-0.05\height}\faGithub\ username} \ $|$ \ 
% \href{https://linkedin.com/in/username}{\raisebox{-0.05\height}\faLinkedin\ username} \ $|$ \ 
% \href{https://mysite.com}{\raisebox{-0.05\height}\faGlobe \ mysite.com} \ $|$ \ 
% \href{mailto:email@mysite.com}{\raisebox{-0.05\height}\faEnvelope \ email@mysite.com} \ $|$ \ 
% \href{tel:+000000000000}{\raisebox{-0.05\height}\faMobile \ +00.00.000.000} \\
% \end{tabularx}

\begin{tabularx}{\linewidth}{@{} C @{}}
\Huge{Bin Xia} \\[7.5pt]
\href{mailto:bxia34@gatech.edu}{\raisebox{-0.05\height}\faEnvelope \ bxia34@gatech.edu} \ $|$ \ 
\href{https://github.com/Xsmos}{\raisebox{-0.05\height}\faGithub\ Xsmos} 
\ $|$ \ 
\href{https://www.linkedin.com/in/xsmos/}{\raisebox{-0.05\height}\faLinkedin\ xsmos}
% \ $|$ \ 
% \href{https://xsmos.github.io/}{\raisebox{-0.05\height}\faGlobe \ xsmos.github.io} 
\\%\ $|$ \ 
% \href{tel:+14705302245}{\raisebox{-0.05\height}\faPhone \ +1 (470)-530-2245} \\
\end{tabularx}

%----------------------------------------------------------------------------------------
% EXPERIENCE SECTIONS
%----------------------------------------------------------------------------------------

\section{Education}
\begin{tabularx}{\linewidth}{@{}l X@{}}	
2021 - present & {Ph.D. Candidate, Physics}, Georgia Institute of Technology\\% \hfill \normalsize (GPA: 4.0/4.0) \\
% 2024 - present 
& M.S., Computational Science \& Engineering, Georgia Institute of Technology \\
& \textit{Advisor: John Wise}\\
% & \textit{GPA: 4/4}\\
% 2019 - 2021 & Unemployed due to extended vacation and the Covid19 \\
% 2019 - 2021 & {After graduating in 2019, I had a half-year gap to prepare for graduate school applications, followed by an 18-month delay due to the pandemic, resuming work in August 2021.}\\
2015 - 2019 & {B.S., Physics with Highest Honor}, Jilin University, China\\
& \textit{Advisor: Coleman Miller}{, the University of Maryland, College Park}\\%\hfill (GPA: 3.95/4.0) \\ 
& \textit{Rank: 1/289, GPA: 3.95/4}\\%\hfill (GPA: 3.95/4.0) \\ 
\end{tabularx}

%----------------------------------------------------------------------------------------
% EXPERIENCE SECTIONS
%----------------------------------------------------------------------------------------

\section{Selected Honors \& Awards}

% \begin{itemize}[leftmargin=*]
\begin{itemize}[leftmargin=0cm]
\setlength{\itemsep}{-5pt}
\item[] Top 10 Distinguished Students Award, Jilin University (0.02\%)\hfill 2019
\item[] President Scholarship, Jilin University ({0.02\%}) \hfill 2019
\item[] Dean Scholarship, College of Physics ({0.5\%})\hfill 2019
\item[] First-class Scholarship, Jilin University\hfill 2019
\item[] China National Scholarship (0.2\%) \hfill({Twice}) 2017 \& 2018
\end{itemize}


% Top 10 Distinguished Students Award, Jilin University (0.02\%)\hfill 2019\\
% President Scholarship, Jilin University ({0.02\%}) \hfill 2019\\
% Dean Scholarship, College of Physics ({0.5\%})\hfill 2019\\
% First-class Scholarship, Jilin University\hfill 2019\\
% China National Scholarship (0.2\%) \hfill({Twice}) 2017 \& 2018
% \begin{tabularx}{\linewidth}{ @{}l r@{} }
% \textbf{Designation} & \hfill Mar 2019 - Jan 2021 \\[3.75pt]
% \multicolumn{2}{@{}X@{}}{
% \begin{minipage}[t]{\linewidth}
%     \begin{itemize}[nosep,after=\strut, leftmargin=1em, itemsep=3pt]
%         \item[--] long long line of blah blah that will wrap when the table fills the column width
%         \item[--] again, long long line of blah blah that will wrap when the table fills the column width but this time even more long long line of blah blah. again, long long line of blah blah that will wrap when the table fills the column width but this time even more long long line of blah blah
%     \end{itemize}
%     \end{minipage}
% }
% \end{tabularx}


%Experience
\section{Selected Research Experience}

\begin{tabularx}{\linewidth}{ @{}l r@{} }
% \textbf{Graduate Research Assistant} & \hfill May 2022 - present \\[3.75pt]
\textbf{{Cosmological 21cm Lightcones Generation Using High-Fidelity Conditional Diffusion}} & \hfill 2024\\
\multicolumn{2}{@{}X@{}}{
\begin{minipage}[t]{\linewidth}
    % \textbf{\textit{Cosmological 21cm Lightcones Generation Using High-Fidelity Conditional Diffusion}}
    \begin{itemize}[nosep,after=\strut, leftmargin=1em, itemsep=3pt]
        \item Developed a cutting-edge conditional diffusion model to generate high-fidelity 3D 21cm brightness temperature lightcones, bridging machine learning with large-scale cosmological data analysis
        \item Overcame computational challenges by employing multi-GPU parallelism, gradient accumulation, mixed precision training, and checkpointing to handle large-scale 3D data, ensuring stable model convergence
        \item Demonstrated exceptional agreement with semi-numerical simulations generated by 21cmFAST in terms of global temperature, power spectrum, and scattering transform coefficients, validating the accuracy and robustness of the model
        \item Tools \& Techniques: PyTorch, high-performance computing, advanced hyperparameter tuning
        % \item Extended the \textit{Accelerated Reionization Era Simulations (ARES)} python package to include the baryon-DM interactions when generating a global 21cm signal model
        % \item Developed an inference program to estimate the mass of DM and the streaming velocity between DM and gas, given a 21cm signal intensity and redshift
        % \item Developed an interpolation scheme to produce an all-sky map of the foreground emission from the Milky Way, including absorption and scattering from intervening gas. This product informs the instrument design for the minimum resolution required for a space-based radio telescope
    \end{itemize}
    \end{minipage}
}
\end{tabularx}

\begin{tabularx}{\linewidth}{ @{}l r@{} }
% \textbf{Graduate Research Assistant} & \hfill May 2022 - present \\[3.75pt]
% \textbf{\textit{Cosmological 21cm Lightcones Generation Using High-Fidelity Conditional Diffusion}}\\
\textbf{{Space-Based Radio Astronomy with Spacecraft Formations}} & \hfill May 2022 - Dec. 2023\\
\multicolumn{2}{@{}X@{}}{
\begin{minipage}[t]{\linewidth}
    % \textbf{\textit{Space-Based Radio Astronomy with Spacecraft Formations}}
    \begin{itemize}[nosep,after=\strut, leftmargin=1em, itemsep=3pt]
        \item Calculated the temperature evolution of baryons and dark matter (DM) during the early universe
        \item Explored the heating effect of the relative velocity between baryons and milli-charged DM
        \item Extended the \textit{Accelerated Reionization Era Simulations (ARES)} python package to include the baryon-DM interactions when generating a global 21cm signal model
        \item Developed an inference program to estimate the mass of DM and the streaming velocity between DM and gas, given a 21cm signal intensity and redshift
        \item Developed an interpolation scheme to produce an all-sky map of the foreground emission from the Milky Way, including absorption and scattering from intervening gas. This product informs the instrument design for the minimum resolution required for a space-based radio telescope
    \end{itemize}
    \end{minipage}
}
\end{tabularx}

\begin{tabularx}{\linewidth}{ @{}l r@{} }
\textbf{Retention of Intermediate-mass Black Holes with Dynamical Friction} & \hfill Jan. 2019 - July 2019 \\[3.75pt]
\multicolumn{2}{@{}X@{}}{
\begin{minipage}[t]{\linewidth}
    \begin{itemize}[nosep,after=\strut, leftmargin=1em, itemsep=3pt]
    % \begin{itemize}[leftmargin=*]
    % \setlength{\itemsep}{-5pt}
    % \item Investigated the map from the parameter space of binary black hole to the gravitational wave kick of the merger remnant of the two progenitor black holes (BHs) by symmetry and spin expansion
    \item Explored the formation mechanisms of intermediate-mass black holes (IMBHs) within globular clusters
    \item Designed numerical programs to estimate the gravitational recoil velocity of binary black hole mergers
    \item Calculated the final recoil of IMBH after black hole merges in a globular cluster with particular distributions of mass ratios, spins, relative spin orientations, and orbital eccentricity
    \item Simulated the retention probability of IMBHs as a function of initial mass by \textit{Monte Carlo methods}
    % \item \textbf{Creatively derived three approximate formulae to conveniently calculate the impact of Chandrasekhar dynamical friction on recoil velocity and analyzed their effects on IMBH retention within globular clusters; reduced program complexity and runtime}
    % \item Evaluated the impact of different recoil formulae, parameters and mass distributions on IMBH retention
    % \item Optimized program for potentially complex BH mass distributions by employing \textit{rejection sampling}
    % \item Improved skills of building theoretical and computational models based on implications of observations, assumptions, and innovations; mastered Python, C and \LaTeX; improved organization and writing skills
    % \end{itemize}
    \end{itemize}
    \end{minipage}
}
\end{tabularx}


\section{Teaching \& Mentoring Experience}

\begin{tabularx}{\linewidth}{ @{}l r@{} }
\textbf{Research Mentor} \\[3.75pt]%& \hfill 2022 - present \\[3.75pt]
\multicolumn{2}{@{}X@{}}{
\begin{minipage}[t]{\linewidth}
    \begin{itemize}[nosep,after=\strut, leftmargin=0em, itemsep=3pt]
        \item[] Vanessa Montgomery \hfill 2022 - 2024
        % \item[] Vanessa Montgomery
    \end{itemize}
    \end{minipage}
}
\end{tabularx}

\begin{tabularx}{\linewidth}{ @{}l r@{} }
\textbf{Teaching Assistant} \\[3.75pt]
\multicolumn{2}{@{}X@{}}{
\begin{minipage}[t]{\linewidth}
    \begin{itemize}[nosep,after=\strut, leftmargin=0em, itemsep=3pt]
    % \begin{itemize}[leftmargin=*]
    % \setlength{\itemsep}{-5pt}
    \item[] Introductory Mechanics Labs (4 sections) \hfill Aug. 2021 - May 2022 
    % \item[] Responsibilities include leading classroom discussions and labs, grading, and holding office hours to provide additional guidance to students.
    \item[] Kentucky-Rutgers-Jilin University International Courses \hfill May - June 2018
    \item[] Atomic Physics, where I edited a \textit{Question \& Answer Manual} for about {200} students \hfill 2017
    \end{itemize}
    \end{minipage}
}
\end{tabularx}

%----------------------------------------------------------------------------------------

%----------------------------------------------------------------------------------------
%	PUBLICATIONS
%----------------------------------------------------------------------------------------
\section{Publications}
% \begin{refsection}[citations.bib]
% \nocite{*}
% \printbibliography[heading=none]
% \end{refsection}

\begin{itemize}[leftmargin=0cm]
\setlength{\itemsep}{-5pt}
% \item[] C. Brummel-Smith, D. R. Skinner, S. Sethuram, J. H. Wise, \textbf{Bin Xia}, K. Taori, 2022, “Inferred galaxy
% properties during Cosmic Dawn from early JWST Photometry Results”. Submitted to MNRAS
% \item[] C. Brummel-Smith, D. R. Skinner, S. Sethuram, J. H Wise, \textbf{Bin Xia}, K. Taori, \href{https://doi.org/10.1093/mnras/stad2569}{Inferred galaxy properties during Cosmic Dawn from early JWST photometry results}, \textit{Monthly Notices of the Royal Astronomical Society}, November 2023
\item[] \textbf{Bin Xia}, J. H. Wise, ``Cosmological 21cm Lightcones Generation Using High-Fidelity Conditional Diffusion''. The Astrophysical Journal. \textit{In Prep}.
\item[] S. Henry, M. Mancini, \textbf{Bin Xia}, J. H. Wise, S. Storm, K. Burkhardt, D. Richardson, G. Badura, V. Montgomery, J. A. Christian, ``Space-Based Radio Astronomy with Spacecraft Formations''. Acta Astronautica. \textit{In Prep}.
\item[] C. Brummel-Smith, D. R. Skinner, S. Sethuram, J. H. Wise, \textbf{Bin Xia}, K. Taori, 2023, “Inferred galaxy
properties during Cosmic Dawn from early JWST Photometry Results”. Monthly Notices of the Royal Astronomical Society, 525, 4405.
\end{itemize}

\section{Academic Activities}

% \begin{itemize}[leftmargin=*]
\begin{itemize}[leftmargin=0cm]
\setlength{\itemsep}{-5pt}
\item[] Peking University \& Tsinghua University visiting student \hfill Aug. 2023  
\item[] Delivered a presentation on How to Be an Outstanding Student at {\textit{TED$\times$JLU}}\hfill Oct. 2019
\item[] Gave Astrophysics lectures for outstanding elementary school students in Jilin Province\hfill 2018 - 2019
\item[] Harbin Institute of Technology-Jilin University Everest Project Exchange\hfill June 2019
\item[] Sun Yat-sen University Fu Lan Physics Festival (Twice)\hfill Dec. 2017, 2018
\item[] Gave a lecture on black holes for Harbin Institute of Technology visiting students\hfill Nov. 2018
\item[] Cambridge-JLU Summer Academic Programme in Computational Physics\hfill Aug. 2018
\item[] Delivered a presentation on How to Achieve Academic Success for about {1000} students\hfill Apr. 2018
\item[] Nagoya University-Jilin University TAQ Class Summer Program\hfill July - Aug. 2017
\end{itemize}
%	SKILLS
%----------------------------------------------------------------------------------------
\section{Skills}
\begin{tabularx}{\linewidth}{@{}l X@{}}
Programming &  \normalsize{Python, Linux/Unix, Bash scripting, Slurm, Git, MATLAB, C, \LaTeX}\\
Machine learning &  \normalsize{Diffusion model, Transformer, Consistency model, Generative adversarial network, Convolutional neural network, Long short-term memory, Random forest}\\
Scientific skills & \normalsize{data analysis, data visualization, data reduction, feature engineering, unstructured data, qualitative data, volumetric data, technical writing, statistics}\\  
\end{tabularx}


\section{Extracurricular Activities}

% \begin{itemize}[leftmargin=*]
\begin{itemize}[leftmargin=0cm]
\setlength{\itemsep}{-5pt}
\item[] Served as a coach of regular trampolines and skiing activities in Changchun\hfill Sept. 2018 - June 2021
\item[] Served as the leader of a game development team; released the video game \textit{JLU Defence}\hfill 2019
\item[] Volunteered at an orphanage in Atlanta, USA\hfill Apr. 2018
\item[] {Leading role} of the film \textit{Reverse}, calling the public to care for people with depression or autism\hfill 2017
\item[] Learning {breakdancing} since 2010; serving as a coach of the University {Street Dance} club since\hfill 2015
\end{itemize}


%Interests/ Keywords/ Summary
% \section{Summary}
% This CV can also be automatically complied and published using GitHub Actions. For details, \href{https://github.com/jitinnair1/autoCV}{click here}.

% %Experience
% \section{Work Experience}

% \begin{tabularx}{\linewidth}{ @{}l r@{} }
% \textbf{Designation} & \hfill Jan 2021 - present \\[3.75pt]
% \multicolumn{2}{@{}X@{}}{long long line of blah blah that will wrap when the table fills the column width long long line of blah blah that will wrap when the table fills the column width long long line of blah blah that will wrap when the table fills the column width long long line of blah blah that will wrap when the table fills the column width}  \\
% \end{tabularx}

% \begin{tabularx}{\linewidth}{ @{}l r@{} }
% \textbf{Designation} & \hfill Mar 2019 - Jan 2021 \\[3.75pt]
% \multicolumn{2}{@{}X@{}}{
% \begin{minipage}[t]{\linewidth}
%     \begin{itemize}[nosep,after=\strut, leftmargin=1em, itemsep=3pt]
%         \item[--] long long line of blah blah that will wrap when the table fills the column width
%         \item[--] again, long long line of blah blah that will wrap when the table fills the column width but this time even more long long line of blah blah. again, long long line of blah blah that will wrap when the table fills the column width but this time even more long long line of blah blah
%     \end{itemize}
%     \end{minipage}
% }
% \end{tabularx}

% %Projects
% \section{Projects}

% \begin{tabularx}{\linewidth}{ @{}l r@{} }
% \textbf{Some Project} & \hfill \href{https://some-link.com}{Link to Demo} \\[3.75pt]
% \multicolumn{2}{@{}X@{}}{long long line of blah blah that will wrap when the table fills the column width long long line of blah blah that will wrap when the table fills the column width long long line of blah blah that will wrap when the table fills the column width long long line of blah blah that will wrap when the table fills the column width}  \\
% \end{tabularx}

% %----------------------------------------------------------------------------------------
% %	EDUCATION
% %----------------------------------------------------------------------------------------
% \section{Education}
% \begin{tabularx}{\linewidth}{@{}l X@{}}	
% 2030 - present & PhD (Subject) at \textbf{University} \hfill \normalsize (GPA: 4.0/4.0) \\

% 2023 - 2027 & Bachelor's Degree at \textbf{College} \hfill (GPA: 4.0/4.0) \\ 

% 2022 & Class 12th Some Board \hfill  (Grades) \\

% 2021 & Class 10th Some Board \hfill  (Grades) \\
% \end{tabularx}

% %----------------------------------------------------------------------------------------
% %	PUBLICATIONS
% %----------------------------------------------------------------------------------------
% \section{Publications}
% \begin{refsection}[citations.bib]
% \nocite{*}
% \printbibliography[heading=none]
% \end{refsection}

% %----------------------------------------------------------------------------------------
% %	SKILLS
% %----------------------------------------------------------------------------------------
% \section{Skills}
% \begin{tabularx}{\linewidth}{@{}l X@{}}
% Some Skills &  \normalsize{This, That, Some of this and that etc.}\\
% Some More Skills  &  \normalsize{Also some more of this, Some more that, And some of this and that etc.}\\  
% \end{tabularx}

\vfill
\center{\footnotesize Last updated: \today}
 
\end{document}
